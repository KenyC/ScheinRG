%plurals_event
%PRECOMPILE COMMAND pdftex -ini -jobname="plurals_event_chap_34" "&pdflatex" mylatexformat.ltx plurals_event_chap_34.tex
\documentclass[english]{article}
\usepackage[T1]{fontenc}
\usepackage[latin1]{inputenc}
\usepackage[sort]{natbib}
\usepackage{stdprmbl}

\usepackage{parskip}

\lingset{interpartskip = -5pt}
\lingset{aboveexskip = 3pt,belowexskip = 3pt}

\definecolor{darkred}{rgb}{0.4,0,0}



\title{Plurals and events (chap. 3)}
\author{Keny Chatain}

\begin{document}
\maketitle

\section*{Intro}
Schein makes two claims which together, will support Neo-Davidsonian separation:

\begin{enumerate}
\item There are no plural entities. Plurals must be handled using set and predicates.\\
\emph{In particular, quantifiers must be sorted in singular quantifiers and plural quantifiers.}
\item Natural language predicates only make reference to pluralities
\end{enumerate}
%
Together, this claim are taken to mean that a natural language event predicate $V$ is equivalent to a separated formula\footnote{These are logical syntax formulas ; the syntax of the intermediate language of representation in indirect semantics}. 

$$V(e, X_1, \ldots, X_n) \leftrightarrow V'(e) \wedge \left( \forall x, \Theta(x, e) \leftrightarrow x\in X_1\right) \ldots $$

In other words, natural language predicates are separated in the \quo{logical syntax} (i.e. the syntax of the intermediate language of representation in indirect semantics)

\section*{Claim I: singulars and plurals must be kept ontologically separate}

The argument consists in exhibiting one quantifier which, despite plural number morphology, must quantify over singularities: the quantifier \emph{few}. This state of affairs

More generally then, it is claimed that quantifiers must sorted in two classes: \textbf{Sg} and \textbf{Pl} (irrespective of their number morphology). This is expected under the view that variables but more difficult to 

\paragraph{If plural entities existed\ldots}
%
Aiming for a contradiction, let us assume that there are indeed plural entities and these may be true or false of the same predicates which singular entities may be true of:

\ex
$\exte{cluster}$ is true of $x$ iff $x$ are clustered
\xe
%
However, quantifiers raise a challenge. Let's look at \emph{few}. If we're only interested in singularities, we may write a definition of it as follows:

\ex \label{naive}
 $\dbb{few}(P)(Q)$ true iff $\#P\cap Q$ is little (compared to $\#P$)
\xe
%
However, this is inadequate if the predicate is true of plural entities. Note that \emph{few} can combine with collective predicates, which can only be true of plural entities:

\ex\label{collab}
Few composers collaborated.
\xe
%
Applying \cref{naive} would lead us to count groups of composers. But intuitively, \emph{few} is still counting single composers in that case, not group of composers. Schein suggests to adapt the denotation of \emph{few} to keep quantification of singularities:

\ex
{\color{darkred}$\dbb{few}(P)(Q)$ true iff $\#\left\lbrace x\ \middle|\ 
\begin{array}{l}
x\text{ is atomic}\\
{\color{darkblue}\exists y, x\prec_{part} y \wedge Q(y) \wedge P(y)}
\end{array}
\right\rbrace$ is little \label{few2}
}\xe
%
This is adequate for \cnextx{collab}. I have distinguished a blue part -the counting of singularities per se- and a red part -the creation of a plural variable that can be fed to collective predicate.
In the sequel, Schein will show that the blue part and the red part come apart, suggesting that not both are core to the meaning of \emph{few}.

\paragraph{Split-scope.} If we insert another quantifier in between the collective predicate and \emph{few}, we see that the red part and the blue part split scope.

\pex
\a Few composers [collaborated on less than 4 operas]
\a 
\textbf{\cnextx{a}'s predicted reading:}\\
 for few composers, there is someone that they collaborated with on less than 4 operas\\
{\color{darkred}red} $\gg$ {\color{darkblue}blue}$\gg$ less than 4
\a \textbf{\cnextx{a}'s actual reading\footnote{Dubbed the \emph{semidistributive} reading?}:}\\
 few composers had less than 4 collaborations\\ 
% In other words, many composers have more than 4 collaborations
{\color{darkred}red} $\gg$ less than 4 $\gg$ {\color{darkblue}blue}
\xe
%
\clastx{b} is what we obtain from plugging in the meaning of  \emph{few} to the bracketed predicate in \clastx{a}. This gives us a very strong reading, where there were collaborations between almost every pair of composers. For instance, the sentence is made false by the existence of a single composer who never collaborated with anyone. 

Rather, the meaning seems to be \clastx{c}, where the creation of a plural group (the blue part) happens below the \quo{less than 4} quantification. Note that this group creation part is also observed in sentences that do not involve \emph{few} such as \cnextx. They allow a singular quantifier to combine with a seemingly collective predicate:

\ex
Hammerstein collaborated on less than 4 operas
\xe
%
All of this suggests that the true meaning of \emph{few} only contains the red part

\ex
$\dbb{few}(P)(Q)$ true iff {\color{darkred}$\#\left\lbrace x\ \middle|\ 
\begin{array}{l}
x\text{ is atomic}\\
P(x)
\end{array}
\right\rbrace$ is little \label{few2}
}\xe
%
This means that \emph{few}, despite number marking, quantifies over singularities.

\paragraph{What will go wrong in general.} 
Schein attempts to give the general shape of the problematic examples.
In Schein's view, there are two classes of quantifiers:

\begin{itemize}
\item \textbf{Class Sg:} quantifiers over singularities 
\item \textbf{Class Pl:} quantifiers over pluralities
\end{itemize}
%
In Schein's view, both can combine with collectives, but Sg quantifiers will need an extra converting step (the blue part). This extra converting step (to be discussed in chap. 8) must take place below event closure, as far as I understand. Thus, a problem will arise whenever a Sg quantifier will try to combine with collective predicate across an event closure. An intermediate $Q$ is inserted to ensure that Sg indeed outscopes event closure. Crucially, $Q$ must not commute with $\exists e$.\footnote{If my reading is correct, Schein confuses non-increasing quantifiers with quantifiers that commute with existentials. The example with \emph{every} to be given below shows an \emph{increasing} quantifier, that does not commute with existentials\ldots} Otherwise, we couldn't tell the precise location of the extra converting step and somebody could choose to package it in the meaning of the quantifier (as we initially did with \emph{few}).

\ex
\Tikzmark{beg}{Sg Quantifier $x$}\hspace{1.5ex} Qy,\hspace{1.5ex} $\exists e,$\hspace{1.5ex} $V(\Tikzmark{end}{Converting(x)(X)})(y)$\\
\DrawArrow{beg}{end}{below}{\small binds} % [tikz options-]beginning-end-position-text[- distance to line]
\xe
%
The following examples illustrate the contrast between a Sg quantifier \emph{few congressmen} and a Pl quantifier \emph{four congressmen} (both marked in plural), using the general schema of \clastx

\pex \textbf{Baseline 1:} \emph{few} can outscope event closure and \emph{every session} when the predicate is distributive
\a Few congressmen attended every session
\a \textbf{Reading:} most congress men missed some session or other (few $\gg$ every session $\gg$ $\exists e,$)
\xe
%
\pex~ \textbf{Baseline 2:} Pl quantifiers can outscope event closure when the predicate is collective
\a Four congressmen met every session
\a \textbf{Reading:} some group of congressmen religiously convened on every session
\xe
%
\pex \textbf{Target:} Sg quantifiers \emph{cannot} outscope event closure when the predicate is collective
\a Few congressmen met every session
\a \textbf{Missing reading:} some group of congresspeople - which were few in number - religiously convened on every session. (?)\footnote{\quo{An illicit interpretation would combine a property of about the action of a plurality across events with semidistributivity}}
\xe
%

\subsection{No escape from thematic roles}

The claim is that any argument to an event predicate must be the whole group of bearers of some thematic roles. For instance, a subject must be the whole set of agents. The argument is that if that weren't enforced, the meaning of \cnextx would be trivialized.

\pex
\a \textbf{Scenario:} \\
Round 1: 18 marbles in 13 slots (first 10 slots have more than 15 marbles in them) \\
Round 2: 10 marbles in 10 slots (= 1 per slot)
\a Fewer than 15 marbles ever fall into exactly ten slots
\a \textbf{Model truth-conditions:} \\
for all events, such that some marbles $x$ fell into exactly 10 slots $\text{fall}(e, x, y)$,
fewer than 15 marbles are such that for some slots, $\text{fall}(e, x, y)$
\xe
%
\clastx{b} is true in \clastx{a}. \clastx{c} will also be true but only if $\text{fall}(e, x, y)$ is assumed to not be true of the marbles that fell in the first ten slots. In other words, $\text{fall}(e, x, y)$ is only true of $x$ if $x$ is the whole experiencer of the falling.

\begin{boxy}{Assumptions}
\begin{enumerate}
\item Round 1 constitute one indecomposable event
\item The predicted TCs are as stated
\end{enumerate}
\end{boxy}
%
The first assumption is anything but clear. 

\end{document}
