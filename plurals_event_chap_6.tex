%&plurals_event
% %PRECOMPILE COMMAND pdftex -ini -jobname="plurals_event_chap_34" "&pdflatex" mylatexformat.ltx plurals_event_chap_34.tex
% \documentclass[english]{article}
% \usepackage[T1]{fontenc}
% \usepackage[latin1]{inputenc}
% \usepackage[sort]{natbib}
% \usepackage{stdprmbl}

% \usepackage{parskip}

% \lingset{interpartskip = -5pt}
% \lingset{aboveexskip = 3pt,belowexskip = 3pt, belowpreambleskip = -3pt}

\endofdump

\definecolor{darkred}{rgb}{0.4,0,0}

\newcommand{\scale}1

\title{Plurals and events (chap. 6) : A Semantics for Plurality and Quantification}
\author{Keny Chatain}

\begin{document}
\maketitle

In this chapter, Schein outlines the assumptions about the syntax/semantics interface.

\section{}

Schein assumes an indirect semantics and an LF level. This means that between the surface syntax and the truth-conditions, there occur at least three translation procedures:

\ex
Surface syntax $\Rightarrow$ LF $\Rightarrow$ Logical formula\footnote{Following May, he calls this part \emph{logical form} and reserves LF to the covert syntax level before it.} $\Rightarrow$ Model-theoretic interpretation
\xe
%
The LF is taken to involve QR to disambiguate quantifier scope, as May intended it.

\paragraph{Essential separation.} Schein wishes to suggest that essential separation of thematic roles is in place in the logical formulas\footnote{There is a reading of him where essential separation happens at LF but that seems crazy to me.}, i.e. that the logical form of \cnextx[a] is the logical form of \cnextx[b]

\pex \label{nomore}
\a No more than 2 detectives found solutions to no more than 5 crimes
\a No more than 2 detectives found solutions to crimes and that was to no more than 5 crimes
\xe
%
This is hard to digest, Schein recognizes, because \clastx{b} involves a non-trivial ellipsis from within a DP:

\ex
That was \sout{a finding of solutions} to no more than 5 crimes.
\xe
%
If this is the reading of \cref{nomore}, then logical formulas is bound to involve non-trivial duplications of material from LF. 

\paragraph{LF to logical formulas} Schein admits that he does not really know what rules of translation achieve this curious mapping. He limits himself to stating constraints on the logical formulas that he wishes his semantics to have. \underline{First,} logical formulas should start with an event quantifier which can be of one of three natures: existential, universal/generic, or anaphoric to a set of events.
He claims that chap. 7 will demonstrate that all subsequent reference to events in the logical formula will be anaphoric.

\ex
\textbf{Logical formula:}
$
\left\lbrace 
\begin{array}{l}
\exists e,\\
\forall e,\\
\iota E,\\
\end{array}
\right\rbrace
\hspace{2ex}
\ldots
\hspace{2ex}
\iota E'
\hspace{2ex}
$
\ldots
\hspace{2ex}
\xe
%
\underline{Second,} he wants to rule out very local scopes for quantifiers in DP positions. In particular, these quantifiers should take scope \emph{a minima} in the VP domain. The reason for this comes from negative quantifiers. Consider the three scope possibilities for \emph{nothing} in \cnextx:

\pex
 Nothing moved. \label{move}
\a   
$\exists e,\ \text{move}(e) \wedge\neg\exists x,\ \textsc{Agent}(e) = x$
\a
$\exists e, \neg\exists x,\ \text{move}(e) \wedge\textsc{Agent}(e) = x$
\a 
$\neg\exists x,\exists e,\ \text{move}(e) \wedge\textsc{Agent}(e) = x$
\xe
%
The intuitive reading of \clastx[a] is guaranteed by giving widest scope to \emph{every}, as in \cnextx[c]. \cnextx[b] is vacuously true, while \cnextx[a] is false under the assumption that moving events have agents/undergoers. Schein notes interestingly, that the problematic readings are not endemic to Davidsonian semantics. In fact, the same problem would arise with time quantifiers.

Regardless, Schein doesn't wish to rule out \clastx[b] immediately because he believes negative events are possible. The following sentences are cases in point:

\pex
\a It made no one move.
\a Fatma saw no one move.
\a Gracefully, no one moved.
\xe
%
The problem is that \cref{move}b is far too weak to be a true \quo{negative event} reading. As long as there is something which is not a moving, it is true. But Schein suggests that the right reading obtains via contextual restriction (\quo{I have nothing to say about the content of [the contextual restriction]}).

If we accept these diagnostics then the only thing to rule out is the local scope of the quantifier ; it suffices to impose that QR of a DP quantfier must land minimally in the VP domain.

\underline{Third,} Schein assumes that QR is obligatory for singular quantifiers and optional for plural quantifiers\footnote{First- and second-order quantifiers in Schein's terminology}. The optionality of QR for plural quantifier corresponds to two different readings: with movement, the reading is distributive ; without movement, the reading is non-distributive. 

\section{Definite descriptions of events}

Schein notes that all singular quantifiers can be translated by an equivalent plural quantifiers. So adopting Schein's plural reference as set reference view, \cnextx[a] and \cnextx[b] are equivalent

\pex
\a 
$\exists x, P(x)$
\a 
$\exists X, |X| = 1\wedge \forall x\in X, P(x) $
\xe
%
What Schein wishes to suggest is that event existence closure is over plurals (aka sets). Most of the times, we are okay with a singular event closure because of the remark above but some cases require plural events:

\ex
Twenty composers collaborated on seven shows.
\xe
%
The point here is that this sentence can be read as talking about multiple and rival collaborations. If this is so, a representation is incorrect as it implies that a big collaboration took place. A plural version is more adequate:

\pex
\a 
$\exists e, \text{collaborate}(e) \wedge \ldots$
\a 
$\exists E,\ \forall e\in E,  \text{collaborate}(e) \wedge \ldots$
\xe
%
(This is quite interesting: this does not seem to be always true of the most natural readings, contrary to what a closure-under-sum would lead us to expect:)

\ex
Four people are collaborating
\xe
%
There is a theoretical motivation for set of events. As explained earlier, Schein wants a logical formula for \cnextx[a] similar to the logical formula for \cnextx[b]:

\pex
\a No more than 2 detectives found solutions to no more than 5 crimes
\a No more than 2 detectives found solutions to crimes and that was to no more than 5 crimes
\xe
%
The problem is that because \emph{no more than 2 detectives} is DE, there may be no event that the first clause of \clastx[b] is true of. If \emph{that} in the second clause denotes an event, the reference may sometimes fail.
%
This problem does not arise if \emph{that} refers to a sets of events: in case no event is picked up by the first clause, the set accessible to reference by \emph{that} can simply be the empty set.

\section{From logical formulas to truth-conditions}

\subsection{Simple version}

Some logical formulas require multiple existentials over events:

\pex
\a Gracefully, every football player pushed the pram up the hill.
\a $\exists e, \text{graceful}(e) \wedge \forall x\in \text{football player}, \exists{e'\prec e},\ $
\xe
%
Schein is worried that if existential over events are freely available\footnotemark, we predict unattested readings:
\footnotetext{Some of these worries might be alleviated if we had a clear LF to logical formulas mapping.}

\pex
I buttered the toast
\a 
$\exists e,\exists e',\exists e'', \text{buttered}(e) \wedge\text{Theme}(e'') = \text{the toast} \wedge\textsc{Agent}(e') = \text{I} $
\a 
I did something ; something was done to the toast ; some buttering happened.
\xe
%
Taking his clue from temporal logic, Schein suggests that we can constrain the model-theoretic interpretation of logical formulas so that the interpretation of low-scope existentials is dependent on the interpretation of high-scope existentials.

\startbox{5cm}
\pex
\a \boxit{\textbf{Logical formula:}} $\exists e,\exists e',\exists e'', \ldots$
\a \boxit{\textbf{Model-theoretic interpretation:}} $\exists e,\exists e'\prec e,\exists e'' \prec e', \ldots$
\xe
%
Schein doesn't develop the system because he needs a further refinement.

\subsection{What about sets of events}

Last subsection only talked about singular events. But we argued that the existential closure was over sets of events. 

Schein proposes an extensive translation of logical formulas into model-theoretic interpretations to deal with this more complex case. I translate his formalism with assignment functions:

\pex \emph{Assignment functions}
\a Assignment functions map variables like $E$ to plural events
\a Model-theoretic interpretation of existentials in logical formulas:\\[2pt]
\quo{$\exists E:\Psi, \Phi$} is true wrt $g$
 \\[2pt]iff\\[2pt]
$\exists g',\ g'=_E g\wedge g'(E)\text{ completely overlaps some part of }g(E)$\\
$\wedge \Psi\text{ is true wrt }g'\wedge\Phi\text{ is true wrt }g'$
\xe
%
There are two important bits in these truth definitions. First, existentials over events in logical formulas shrink a previous value of $g(E)$. This evacuates the problem of multiple existentials\footnotemark:
\footnotetext{
Not quite: the problematic reading with multiple existentials can be emulated using existentials with different variables. So some constraint on logical formula must impose that they are all the same.
}

\pex
I buttered the toast
\a 
$\exists E,\exists E',\exists E'', \textsc{Agent}(E) = \text{I}\wedge\text{Theme}(E') = \text{the toast}\wedge\text{buttered}(E'') $
\a 
I did something. One of the things I did was done to the toast. One of the things I did to the toast was a buttering
\xe
%
Second, the notion of complete overlap. This is a mereological notion not a set-theoretic notion. Intuitively, two set of events completely overlap one another if their events are made from the same parts. 

\begin{boxy}{Complete overlap.} 
$E$ and $E'$ completely overlap one another \\
iff every event that overlaps an event of $E$ overlaps an event of $E'$ and vice-versa
\end{boxy}
%
The effects of using overlap are not directly appreciable with the existentials but with thematic roles:

\ex\label{theme_role}
Model-theoretic interpretation of thematic roles in logical formulas:\\[2pt]
\quo{$\textsc{Agent}(E) = X$} is true wrt $g$
 \\[2pt]iff\\[2pt]
$\exists g',\ g'=_E g\wedge g'(E)\text{ completely overlaps }g(E)$\\
$\wedge \forall e\in g'(E), \exists x\in X, x\text{ is an agent of }e$\\
$\wedge \forall x\in X, \exists e\in g'(E), x\text{ is an agent of }e$\\
\xe
%
Every time we incorporate an argument into the event $E$, we are allowed to substitute $E$ for a completely overlapping substitute. Here is an example where this plays a role:

\pex
\a 
\textbf{Context:}
\emph{
Brotherhoods of ninja turtles eat separately from each other, each on their own table. 
Within a brotherhood, the turtles share slices of pizza.
A single brotherhood never eats a whole pizza.
}
\a \label{exam}
A hundred and seventeen turtles shared twenty-three pizzas.
\xe
%
From the verb \emph{shared}, we know that the sentence describes a set of sharing events, presumably one for each table. \quo{twenty-three pizzas}  describes a set of twenty-three pizzas. However, the themes of the event of the turtle's sharing do not form a set of twenty-three pizzas. Rather, they form a set of slices, because the turtles' brotherhood eat by the slice. It would seem then that the sentence is false.

While the set of all sharings does not completely overlap with one sharing (as the turtles are not sharing across brotherhoods), it completely overlaps with a eating. And that eating has all the pizzas as a theme. Having introduced complete overlap in our truth definitions of logical formulas, we may substitute the set of sharing events with the whole eating event when evaluating whether 23 pizzas were the theme of the event.

\subsection{Should we be worried about overlap?}

\paragraph{A familiar example}
The notion of \emph{overlap} is theoretical and may not match intuitions. Schein reiterates this points with the following example from Chap. 5:

\pex
\a The Carnegie Hall sits opposite the Carnegie Deli.
\a The Carnegie Deli sits opposite the Carnegie Hall.
\xe
%
In Chap. 5, Schein showed that the sentence in \clastx cannot be taken to describe the same event/state, despite what common intuitions might suggest. Here we see that they should not even describe overlapping events. Otherwise, our truth definitions for thematic roles would allow us to replace the holders of the thematic roles from one event with the holders from another (just as we replaces pizza slices with pizzas in \cref{exam}):

\ex
The Carnegie Hall sit opposite the Carnegie Deli.
\xe
%
Schein concludes that spatiotemporal complete overlap is not a sufficient condition for two events to overlap. He adds that maybe, for two states of sitting to overlap, it is furthermore required that their orientation be the same.

\paragraph{An unfamiliar example}
There are more worrying cases. Consider the following case: a psychologist compares two groups of subjects along some dimension. 

\pex
\a The nailbiters outscored the bedwetters. (\emph{sic})
\a \textbf{Logical Formula:}\\
$\exists E, \textsc{Agent}(E)=\text{the nailbiters} \wedge \textsc{Theme}(E)=\text{the bedwetters} \wedge \text{outscore}(E)$ \label{lf2}
\xe
%
(Here, Schein seems to understand \emph{outscore} as true if the sum of scores from one group is greater than the sum of scores from the other group.) All subjects obtained the same score but there are 3 bedwetters and just two nailbiters. 

Schein imagines that there is an event of scoring $e$ which includes the nailbiters and bedwetters. That event certainly includes all the nailbiters and bedwetters and is an event of \emph{outscoring} ; however, the holders of this state are not the nailbiters as \clastx seems to require but the bedwetters, so \clastx is false.


Yet, the substitution of completely overlapping events in thematic positions seems like it could make the sentence true. In particular, it may be thought that $e$ overlaps
with three events $e_1$, $e_2$ and $e_3$: one where the two nailbiters outscore the first bedwetter, one where they outscore the second bedwetter, one where they outscore the third bedwetter. If $e_1$, $e_2$ and $e_3$ completely overlap $e$, then we predict that we could make \cref{lf2} true.

Schein concludes that $e$ must not be treated as completely overlapping $e_1$, $e_2$ and $e_3$.

I've followed Schein's reasoning so far, but the problem, as he ends up noticing, is not so much about the complete overlap that we introduce in the truth definition. The reasoning was made under the assumption that $E$ in  \cref{lf2} is $e$, but nothing prevents us from considering $E=\left\lbrace e_1, e_2, e_3 \right\rbrace$. In which case the sentence is made true by the same reasoning

Schein ends up claiming that the sentence is ambiguous. On one reading, the one described in \cref{lf2}, it is indeed true. However, there is a \quo{salient} reading where it talks about only 1 event:

\ex
$\exists E, \underline{|E| = 1\wedge} \textsc{Agent}(E)=\text{the nailbiters} \wedge \textsc{Theme}(E)=\text{the bedwetters} \wedge \text{outscore}(E)$ \label{lf2}
\xe
%
Under that reading, the sentence can only be talking about ${e}$, ${e_1}$, ${e_2}$, ${e_3}$ but not plural events. (If this is true, notice that we lose the argument that $e_1$, $e_2$ and $e_3$ do not completely overlap $e$) 

Schein tries to motivate this ambiguity by exhibiting predicates which seem to impose the plural event to be a singularity like \emph{dense}.

\pex
\a The elms are clustered ($\checkmark$ several clusters)
\a The elms are dense ($\checkmark$ several dense spots)
\xe
%
Schein suggests before retracting that this is related to the following fact:

\pex
\a The elms are all clustered.
\a *The elms are all dense.
\xe
%






\end{document}
